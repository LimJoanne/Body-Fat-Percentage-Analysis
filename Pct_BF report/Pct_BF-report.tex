\documentclass[letterpaper,9pt,twocolumn,twoside,]{pinp}

%% Some pieces required from the pandoc template
\providecommand{\tightlist}{%
  \setlength{\itemsep}{0pt}\setlength{\parskip}{0pt}}

% Use the lineno option to display guide line numbers if required.
% Note that the use of elements such as single-column equations
% may affect the guide line number alignment.

\usepackage[T1]{fontenc}
\usepackage[utf8]{inputenc}

% pinp change: the geometry package layout settings need to be set here, not in pinp.cls
\geometry{layoutsize={0.95588\paperwidth,0.98864\paperheight},%
  layouthoffset=0.02206\paperwidth, layoutvoffset=0.00568\paperheight}

\definecolor{pinpblue}{HTML}{185FAF}  % imagecolorpicker on blue for new R logo
\definecolor{pnasbluetext}{RGB}{101,0,0} %



\title{013E01 Executive summary}

\author[a]{Lim Joanne}
\author[a,b]{John Fu}


\setcounter{secnumdepth}{0}

% Please give the surname of the lead author for the running footer
\leadauthor{Joanne and John}

% Keywords are not mandatory, but authors are strongly encouraged to provide them. If provided, please include two to five keywords, separated by the pipe symbol, e.g:
 

\begin{abstract}
Your abstract will be typeset here, and used by default a visually
distinctive font. An abstract should explain to the general reader the
major contributions of the article.
\end{abstract}

\dates{This version was compiled on \today} 


% initially we use doi so keep for backwards compatibility
% new name is doi_footer


\begin{document}

% Optional adjustment to line up main text (after abstract) of first page with line numbers, when using both lineno and twocolumn options.
% You should only change this length when you've finalised the article contents.
\verticaladjustment{-2pt}

\maketitle
\thispagestyle{firststyle}
\ifthenelse{\boolean{shortarticle}}{\ifthenelse{\boolean{singlecolumn}}{\abscontentformatted}{\abscontent}}{}

% If your first paragraph (i.e. with the \dropcap) contains a list environment (quote, quotation, theorem, definition, enumerate, itemize...), the line after the list may have some extra indentation. If this is the case, add \parshape=0 to the end of the list environment.


\hypertarget{data-set.}{%
\subsection{\texorpdfstring{\texttt{Data\ set.}}{Data set.}}\label{data-set.}}

The data set we use in our analysis was originally collected by SOCR to
estimate the percentage of body fat determined by underwater weighing
and various body circumference measurements for 250 men.

There are 16 variables included in the data and the key dependent
variable for this analysis is \texttt{Pct.BF} - a variable measuring
percentage body fat of men.

\hypertarget{analysis.}{%
\subsection{\texorpdfstring{\texttt{Analysis.}}{Analysis.}}\label{analysis.}}

Our modeling began by first eliminating the `density' variable from the
data set due to its challenging nature of measurement in real-world
settings and limited practical utility. Following this, we employed both
backward and forward selection model based on the AIC to identify the
optimal predictors for body fat percentage.

For backward selection, we firstly run a full regression model which
contains all the explanatory variables in our data set. Then we try to
drop variables to lower the AIC, and this narrowed down to eight key
predictors: Age, Height, Neck, Abdomen, Hip Thigh, Forearm and Wrist.

Conversely, the forward selection began without any predictors,
incrementally adding the most statistically significant ones. This
process ended up including six predictors: Waist, Weight, Wrist, Bicep,
Age and Thigh.

In conclusion, backward selection model appears to be more compelling
with its slightly higher adjusted R-squared value and lower AIC value.

\hypertarget{assumptions.}{%
\subsection{\texorpdfstring{\texttt{Assumptions.}}{Assumptions.}}\label{assumptions.}}

Before proceed to analyse our models results, we must first ensure that
all our assumptions are satisfied. The key assumptions for our model
are: 1. Linearity - the relationship between Y and x is linear. 2.
Independence - all the errors are independent of each other. 3.
Homoskedasticity - the errors have constant variance. 4. Normality - the
error follows a normal distribution.

By looking at appendix \_\_, we can determine if our assumptions are
met. Since there is no obvious pattern (e.g.~no smiley or frowny face)
in the residual vs fitted values plot, therefore the linearity
assumption is met. The residuals don't appear to be fanning out or
changing their variability over the range ofthe fitted values so the
constant error variance assumption is met, and thus the Homoskedasticity
assumption is met. Also in the QQ plot, apart from the top 6 or so
points, the majority of points lie quite close to the line in the QQ
plot. Hence, the normality assumption for the residuals is reasonably
well satisfied. Additionally, we have quite large sample size so we can
also rely on the central limit theorem to give us approximately valid
inferences. Lastly, the independence of error terms is crucial and
typically addressed during the initial phases of experimental design,
i.e.~\textbf{before data collection}. Each variable is designed to
maintain its independence and since each observation doesn't inherently
influence another, we can conclude that they are independent of each
other.

Therefore given all our assumptions are met, our multiple linear
regression model can be reliably analysed.

\hypertarget{results.}{%
\subsection{\texorpdfstring{\texttt{Results.}}{Results.}}\label{results.}}

Our final model is:

\[\widehat{Pct.BF} = 5.04 + 0.0726Age - 0.268Height - 0.451Neck + 0.822Abdomen - 0.195Hip + 0.224Thigh + 0.295Forearm - 1.731Wrist\]

Our full model has 15 variables and an R-squared of 0.737. However, in
our new simplified model, we dropped 7 variables and obtained an
R-squared of 0.739 which is slightly higher than the full model. The
in-sample performance of the final model provided an R-squared that
shows that approximately 73.9\% of the total variability in Pct.BF is
explained by the explanatory variables.

We used 10-fold cross validation to measure out-of-sample performance.
From the output we are able to see that in the full model, it has a RMSE
of 4.346 and MAE of 3.597. Conversely, the simplified model has a RMSE
of 4.263 and MAE of 3.508. Thus we can see that the simplified model
outperforms the full model.

Making all the explanatory variables equal to zero, on average the
predicted percentage body fat is 5.04\%. Holding other variables
constant, a year increase in age will leads to a 0.07 increase in body
fat percentage. Holding all other variables constant, a 1cm increase in
Height, on average would have a predicted decrease in body fat
percentage by 0.268\%. Holding other variables constant, a 1cm increase
in Neck circumference, on average body fat percentage will decrease by
0.451\%. Holding other variables constant, a 1cm increase in Abdomen
circumference will lead to 0.822\% increase in body fat percentage.
Holding all variables constant, a 1cm increase in Hip circumference, on
average would have a predicted decrease in body fat percentage by
0.195\%. Holding other variables constant, a 1cm increase in Thigh
circumference will lead to 0.224\% increase in body fat percentage.
Holding other variables constant, a 1cm increase in Forearm
circumference will lead to 0.295\% increase in body fat percentage.
Holding other variables constant, a 1cm increase in Wrist circumference,
on average body fat percentage will decrease by 1.731\%.

%\showmatmethods





\end{document}
